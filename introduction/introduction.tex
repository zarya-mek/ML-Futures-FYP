\chapter{Introduction}
Financial markets play an essential role in the prosperity of the global economy. They propel the allocation of resources and creation of liquidity by providing a stage for people to trade financial instruments such as stocks, bonds and commodities. Therefore, strong financial markets are an indicator of a healthy economy. Hence, there is a constant need to improve the strength of the financial markets.\\ \\
As a result, financial time-series forecasting has not only become a top choice amongst practitioners, but is also of growing interest in the artificial intelligence industry. The acceleration of machine learning prediction models in the last decade has propelled the employment of machine learning algorithms to forecast financial time-series, in contrast to utilizing classical statistical methods. \\


\section{Motivation}
The vast majority of time-series forecasting in commodity markets is focused on forecasting futures prices. Classical time-series forecasting methods such as Autoregression (AR), Moving Average (MA), ARMA, Autoregressive Integrated Moving Average (ARIMA) and various other versions have been employed to forecast future data from historical data. More recently, linear regression, ensemble methods and neural networks have all been employed for financial time-series forecasting \cite{nielsen2019practical, lazzeri2020machine} .  Asset portfolio managers aim to generate alpha, or essentially `beat the market' by curating and diversifying investment portfolios. However, as financial markets mature, the stagnation of growth and lack of innovation makes alpha generation increasingly challenging.  \\

Although the theory of classical econometrics is academically the most developed, combining regression methods with feature selection and various `tricks of the trade' are favoured by practitioners. 

\section{Objectives and Challenges}
The overall single objective of this project is to investigate the relative efficacy of machine learning backed futures trading strategies and collate a recommended set of algorithms that achieves a satisfactory and reasonable degree of alpha generation. \\ \\
As a first step, our objective is to first implement, then carry out an in-depth analysis and evaluation of several futures trading strategies that are driven by machine learning algorithms. To be able to recommend an algorithm, it is vital to evaluate strategies using certain specific statistical metrics and plots to gain a clear picture of the performance of each strategy. In the subsequent step, our objective is to compare these future strategies with the classical strategies.  In order to do that, we would compare such futures trading strategies with classical strategies on the basis of successful alpha generation. \\ \\
The evaluation of trading strategies requires careful consideration of many factors that influence results. It is mentioned in \cite{bailey2014deflated}, backtesting of trading strategies are prone to phenomena such as overfitting and look-ahead bias and data-snooping in \cite{daniel2009look}. These facts can lead to the acceptance of trading strategies that provide desirable results during backtests, but perform sub-optimally in practice. Deciding on the backtesting methodology requires rigorous consideration in order to remove bias and provide convincing results. \\ \\
Additionally, machine learning models are prone to underfitting and overfitting on training data, resulting in poor accuracy and efficiency during prediction tasks. These can be avoided using well-established methods such as \hyperref[l1l2]{L1/L2 regularization}.

